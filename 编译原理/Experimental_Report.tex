\documentclass[supercite]{Experimental_Report}

\title{\ }
\school{计算机科学与技术学院}
\classnum{本硕博2001班}
\stunum{U202015630}
\author{李茗畦}
\instructor{杨茂林}
\usepackage{zhnumber} % change section number to chinese
\renewcommand\thesection{\arabic{section}}
\renewcommand \thesubsection {\arabic{section}.\arabic{subsection}}
\usepackage{tikz}
\usetikzlibrary{mindmap,trees}
\usepackage{algorithm, multirow}
\usepackage{algpseudocode}
\usepackage{amsmath}
\usepackage{amsthm}
\usepackage{framed}
\usepackage{multirow}
\usepackage{multicol}
\usepackage{arydshln}
\usepackage{mathtools}
\usepackage{subcaption}
\usepackage{amsmath, amssymb, amsfonts}
\usepackage{tabu}
\usepackage{xltxtra} %提供了针对XeTeX的改进并且加入了XeTeX的LOGO, 自动调用xunicode宏包(提供Unicode字符宏)
\usepackage{bm}
\usepackage{longtable}
\usepackage{tikz}
\usepackage{pdfpages}
\usepackage{tikzscale}
\usepackage{pgfplots}
\usepackage{graphicx}
\usepackage{listings}
\usepackage{titlesec}
\usepackage{fontspec}
\usepackage{xcolor}      %代码着色宏包
\usepackage{CJK}         %显示中文宏包
\usepackage{pifont}
\usepackage{array}
\usepackage{booktabs}





\pgfplotsset{compat=1.16}

\newcommand{\cfig}[3]{
  \begin{figure}[htb]
    \centering
    \includegraphics[width=#2\textwidth]{images/#1.tikz}
    \caption{#3}
    \label{fig:#1}
  \end{figure}
}

\newcommand{\sfig}[3]{
  \begin{subfigure}[b]{#2\textwidth}
    \includegraphics[width=\textwidth]{images/#1.tikz}
    \caption{#3}
    \label{fig:#1}
  \end{subfigure}
}

\newcommand{\xfig}[3]{
  \begin{figure}[htb]
    \centering
    #3
    \caption{#2}
    \label{fig:#1}
  \end{figure}
}

\newcommand{\rfig}[1]{\autoref{fig:#1}}
\newcommand{\ralg}[1]{\autoref{alg:#1}}
\newcommand{\rthm}[1]{\autoref{thm:#1}}
\newcommand{\rlem}[1]{\autoref{lem:#1}}
\newcommand{\reqn}[1]{\autoref{eqn:#1}}
\newcommand{\rtbl}[1]{\autoref{tbl:#1}}

\algnewcommand\Null{\textsc{null }}
\algnewcommand\algorithmicinput{\textbf{Input:}}
\algnewcommand\Input{\item[\algorithmicinput]}
\algnewcommand\algorithmicoutput{\textbf{Output:}}
\algnewcommand\Output{\item[\algorithmicoutput]}
\algnewcommand\algorithmicbreak{\textbf{break}}
\algnewcommand\Break{\algorithmicbreak}
\algnewcommand\algorithmiccontinue{\textbf{continue}}
\algnewcommand\Continue{\algorithmiccontinue}
\algnewcommand{\LeftCom}[1]{\State $\triangleright$ #1}


\newtheorem{thm}{定理}[section]
\newtheorem{lem}{引理}[section]
\colorlet{shadecolor}{black!15}

\theoremstyle{definition}
\newtheorem{alg}{算法}[section]
\def\thmautorefname~#1\null{定理~#1~\null}
\def\lemautorefname~#1\null{引理~#1~\null}
\def\algautorefname~#1\null{算法~#1~\null}
\usepackage{xcolor}
\usepackage{listings}
\usepackage[skins]{tcolorbox}


\tcbuselibrary{breakable}
\newtcolorbox{abox}[1][]{enhanced,
	fonttitle = \heiti \large \bfseries,% 标题字体设置
	colbacktitle =black,% 标题背景颜色
	coltitle=white,% 标题字体颜色
	% halign title=center,% 标题对齐方式
	attach boxed title to top left = {yshift = -5pt},
	%	将标题以box 形式放在文本框左上角,并向下移动5pt	
	%----------正文字体--------
	fontupper = \kaishu,%正文字体设置
	%colback = white,%正文背景颜色设置
	%---------边框设置---------
	arc = 4pt,%正文边框边角弧度
	%	boxed title style={arc=0pt},%标题框边角弧度
	colframe=black,%边框颜色设置
	toprule  = 1pt,%取消上边框
	rightrule = 1pt,%取消下边框
	%---------调整字体位置------
	top = 10pt,%增大正文文本与上边框距离
	%---------设置标题为默认参数,不影响其他参数的设置
	title=#1,
	breakable
}
\newcommand{\tabincell}[2]{\begin{tabular}{@{}#1@{}}#2\end{tabular}}

%修改第三级标题
\usepackage{natbib}
\newcommand{\upcite}[1]{{\setcitestyle{square,super}\cite{#1}}}

\lstset{
    basicstyle=\ttfamily\fontseries{m},  % 将字体变细
    breaklines=true,
    lineskip=-0.5pt,
    columns=fullflexible,  % 配合keepspaces一起使用,保留空格
    %行号
    numbers=left,
    rulesepcolor=\color{red!20!green!20!blue!20},
    escapeinside=``,
    xleftmargin=2em,xrightmargin=2em, aboveskip=1em,
    %背景框
    framexleftmargin=1.5mm,
    frame=single,
    %背景色
    % backgroundcolor=\color[RGB]{245,245,244},
    %样式
    keywordstyle=\color{blue}\bfseries,
    identifierstyle=\ttfamily\color{black},
    numberstyle=\color[RGB]{0,192,192},
    commentstyle=\it\color[RGB]{96,96,96},
    stringstyle=\rmfamily\slshape\color[RGB]{128,0,0},
    %显示空格
    showstringspaces=false
}

\setcounter{tocdepth}{2}  % 设置目录只显示一级标题和二级标题

\begin{document}
\maketitle

\newpage
\thispagestyle{empty}
\mbox{}
\clearpage


\begin{figure}[H]
	\begin{center}
		\includegraphics[scale=1.30]{./images/shengming.jpg}
		\label{fig1-1}
	\end{center}
\end{figure}

\newpage
\thispagestyle{empty}
\mbox{}
\clearpage


\tableofcontents

\newpage
\thispagestyle{empty}
\mbox{}
\clearpage


\section{编译工具链的使用}
\subsection{实验任务}
本节实验需要完成以下任务:
\begin{enumerate}
  \item 编译工具链的使用;
  \item Sysy语言及运行时库;
  \item 目标平台riscv64的汇编语言;
\end{enumerate}

本次实验选择的是目标平台riscv64的汇编语言。
\subsection{实验实现}
\subsubsection{编译工具链的使用}
使用如下命令使用gcc编译器将alibaba.c和def-test.c编译生成命名为def-test的可执行文件,使用-D BILIBILI添加预处理语句\#define BILIBILI
\begin{center}
\begin{lstlisting}[language=bash]
gcc -D BILIBILI alibaba.c def-test.c -o def-test
\end{lstlisting}
\end{center}

使用以下命令使用clang编译器将bar.c编译成armv7汇编代码。其中-O2表示编译器对程序的优化级别,-S表示用于生成汇编代码而不是可执行文件,-target armv7-linux-gnueabihf设置目标平台为ARMv7指令集的Linux操作系统。

\begin{center}
\begin{lstlisting}[language=bash]
clang -S -O2 -target armv7-linux-gnueabihf  bar.c -o bar.clang.arm.s
\end{lstlisting}
\end{center}

使用以下命令使用arm-linux-gnueabihf-gcc将iplusf.c编译成arm汇编代码iplusf.arm.s,之后使用arm-linux-gnueabihf-gcc将iplusf.arm.s连接SysY2022的运行时库sylib.a生成可执行文件iplusf.arm,并在qemu-arm上运行该可执行文件。

\begin{center}
\begin{lstlisting}[language=bash]
arm-linux-gnueabihf-gcc -S iplusf.c -o iplusf.arm.s
arm-linux-gnueabihf-gcc iplusf.arm.s sylib.a -o iplusf.arm
qemu-arm -L /usr/arm-linux-gnueabihf/ ./iplusf.arm\end{lstlisting}
\end{center}

最后要求编写一个Makefile,将main.cc和helloworld.cc编译为文件名为helloworld的可执行文件,其中头文件目录为./include,Makefile如下所示,其中-I用于指定头文件目录。
\begin{center}
\begin{lstlisting}[language=bash]
objects:=main.o helloworld.o
CC:=clang++
CFLAGS:=-O2
INCLUDE:=./include
helloworld:$(objects)
	$(CC) $(CFLAGS) $(objects) -o helloworld
main.o:main.cc
	$(CC) -c -I $(INCLUDE) $(CFLAGS) main.cc -o main.o
helloworld.o:helloworld.cc
	$(CC) -c -I $(INCLUDE) $(CFLAGS) helloworld.cc -o helloworld.o
\end{lstlisting}
\end{center}
\subsubsection{Sysy语言及其运行时库}
本实验较简单,需要注意的就是Sysy是c语言的一个子集,不能使用Sysy不支持的语法。
\subsubsection{目标平台-RISCV}
本实验也较简单,因此不再详细阐述实现过程。
\newpage
\section{词法分析}
\subsection{实验任务}
本实验的任务为实现Sysy词法分析器,我实现的是基于flex的Sysy词法分析器(C语言实现)。需要完成识别标识符ID、int型字面量INT\_LIT和float型字面量FLOAT\_LIT。

\subsection{词法分析器的实现}
首先声明以下一系列规则,分别用于匹配单个数字、单个符号、单个非零十进制数、单个非零八进制数、单个十六进制数和十六进制的前缀"0x"和"0X"。
\begin{lstlisting}[float=htb]
DIGIT [0-9]
LETTER [A-Za-z_]
NOZERO_DIGIT [1-9]
OCTAL_DIGIT [0-7]
HEXADECIMAL_DIGIT [0-9a-fA-F]
HEXADECIMAL_PREFIX "0X"|"0x"
\end{lstlisting}

以下规则分别用于匹配十进制常数、八进制常数、十六进制常数和int型常量。其中十进制常数需要保证第一个数字是非零的十进制数,八进制常数需要保证第一个数是0,十六进制的常数需要保证最开始是十六进制的前缀。int型常量就是十进制、八进制或者十六进制常数。
\begin{lstlisting}[float=htb]
DECIMAL_CONST {NOZERO_DIGIT}{DIGIT}*
OCTAL_CONST 0{OCTAL_DIGIT}*
HEXADECIMAL_CONST {HEXADECIMAL_PREFIX}{HEXADECIMAL_DIGIT}*
INTEGER_CONST {DECIMAL_CONST}|{OCTAL_CONST}|{HEXADECIMAL_CONST}
\end{lstlisting}

使用以下规则匹配float型常量。其中EXPONENT用于匹配浮点数中的指数部分。
\begin{lstlisting}[float=htb]
EXPONENT [eE][-+]?{DIGIT}+
FLOAT_CONST ({DIGIT}+"."{DIGIT}*{EXPONENT}?|{DIGIT}*"."{DIGIT}+{EXPONENT}?|{DIGIT}+{EXPONENT})[Ff]?
\end{lstlisting}

使用以下规则匹配标识符。标识符由字母或者符号组成,第一个字符需要保证是符号。
\begin{lstlisting}[]
IDENTIFIER {LETTER}({LETTER}|{DIGIT})*
\end{lstlisting}
\newpage
\section{语法分析}
\subsection{实验任务}
在给出的语法分析器框架的基础上,实现一个Sysy语言的语法分析。我实现的是一个基于flex/bison的语法分析器(C语言实现)。需要完成的语义规则包括赋值语句、if-else语句、while语句、break语句、continue语句和return语句等。
\subsection{语法分析器的实现}
本实验中我们只需要调用new\_node构造节点即可。实现代码如下。需要注意的是new\_node函数的left、mid和right参数,当只有两个子结点时,需要使用left和right节点,只有一个子结点时使用的是right节点。

\begin{lstlisting}[float=htb]
Stmt :  LVal ASSIGN Exp SEMICOLON { $$ = new_node(Stmt,$1,NULL,$3,AssignStmt,0,NULL,NonType); }
    | SEMICOLON { $$=NULL; }
    | Exp SEMICOLON { $$ = new_node(Stmt,NULL,NULL,$1,ExpStmt,0,NULL,NonType); }
    | Block { $$ = new_node(Stmt,NULL,NULL,$1,Block,0,NULL,NonType); }
    | IF LP Cond RP Stmt { $$ = new_node(Stmt,$3,NULL,$5,IfStmt,0,NULL,NonType); }
    | IF LP Cond RP Stmt ELSE Stmt { $$ = new_node(Stmt,$3,$5,$7,IfElseStmt,0,NULL,NonType); }
    | WHILE LP Cond RP Stmt { $$ = new_node(Stmt,$3,NULL,$5,WhileStmt,0,NULL,NonType); }
    | BREAK SEMICOLON { $$ = new_node(Stmt,NULL,NULL,NULL,BreakStmt,0,NULL,NonType); }
    | CONTINUE SEMICOLON { $$ = new_node(Stmt,NULL,NULL,NULL,ContinueStmt,0,NULL,NonType); }
    | RETURN SEMICOLON { $$ = new_node(Stmt,NULL,NULL,NULL,BlankReturnStmt,0,NULL,NonType); }
    | RETURN Exp SEMICOLON { $$ = new_node(Stmt,NULL,NULL,$2,ReturnStmt,0,NULL,NonType); };
\end{lstlisting}
\newpage
\section{中间代码生成}
\subsection{实验任务}
在给出的中间代码生成器框架基础上完成LLVM IR中间代码的生成,将Sysy语言程序翻译成LLVM IR中间代码。我们需要完成的是GenIR的visit方法中的ASS分支。
\subsection{中间代码生成器的实现}
ASS类型的结点是赋值语句,它有两个子节点:lVal和exp。其语义是将exp的值存储到代表左值的变量地址。首先将requireLVal赋值为true,表示当前的lVal是赋值语句左值,不是表达式。接着调用lVal的accept方法取出表达式的左部,之后再用同样的方法取出表达式的右部。需要注意的是,我们需要检查赋值语句的左值和右值的类型是否匹配,两个操作数必须是同一类型。当左值是int32类型而右值是float类型时,需要将右值转换为int32类型;当左值是float类型而右值是int32类型时,需要将右值转换为float类型。最后调用IRStmtBuilder::create\_store(Value *val, Value *ptr)方法生成store指令。该部分的实现代码如下所示。

\begin{lstlisting}[float=htb,language=C]
requireLVal=true;                       // 当前lVal是赋值语句左部
ast.lVal->accept(*this);                // 取出左部
auto left_val=recentVal;
ast.exp->accept(*this);                 // 取出右部
auto right_val=recentVal;
if(left_val->type_->tid_==Type::TypeID::IntegerTyID && right_val->type_->tid_==Type::TypeID::FloatTyID)
    right_val=builder->create_fptosi(right_val,left_val->type_);
else if(left_val->type_->tid_==Type::TypeID::IntegerTyID && right_val->type_->tid_==Type::TypeID::FloatTyID)
    right_val=builder->create_fptosi(right_val,left_val->type_);
builder->create_store(right_val,left_val);
\end{lstlisting}

\newpage
\section{目标代码生成}
\subsection{实验任务}
在给出的代码框架基础上,将LLVM IR中间代码翻译成指定平台的目标代码。我实现的是生成平台为RISCV64的目标代码。

\subsection{目标代码生成器的实现}
首先需要指定目标三元组。在本实验中将目标平台设置为"riscv64-unknown-elf",并初始化所有用来生成目标代码的目标。之后调用TargetRegistry::lookupTarget 函数,通过刚刚得到的目标三元组得到一个对应的目标。
\begin{lstlisting}[float=htb,language=C]
auto target_triple = "riscv64-unknown-elf";
module->setTargetTriple(target_triple);
InitializeAllTargetInfos();
InitializeAllTargets();
InitializeAllTargetMCs();
InitializeAllAsmParsers();
InitializeAllAsmPrinters();
\end{lstlisting}

接下来需要指定目标机器,我们只需要使用通用CPU。

然后我们生成目标代码。调用getGenFilename()函数,获得要写入的目标代码文件名filename。接下来实例化一个raw\_fd\_ostream类的对象dest,其中Flags成员置sys::fs::OF\_None,实例化legacy::PassManager类的对象pass。然后为file\_type赋初值。该部分代码如下所示。
\begin{lstlisting}[language=C]
std::string filename=getGenFilename(ir_filename,gen_filetype);
std::error_code EC;
auto dest=raw_fd_ostream(filename, EC, sys::fs::OF_None);
if(EC){
    errs()<<"raw_fd_ostream error";
    return 1;
}
legacy::PassManager pass;
auto filetype=gen_filetype;
if (TheTargetMachine->addPassesToEmitFile(pass, dest, nullptr,filetype)) {
    errs() << "TheTargetMachine can't emit a file of this type";
    return 1;
}
pass.run(*module);
dest.flush();
\end{lstlisting}

\newpage
\section{总结}
本次课程的实验主要包括词法分析、语法分析、语义分析、中间代码生成和目标代码生成等几部分。通过这次实验,我对编译原理中的词法分析器、语法分析器等有了一个基本的理解。对编译原理理论课程中学习到的知识进行了时间,加深了对理论课程知识的理解和掌握。


\end{document}
